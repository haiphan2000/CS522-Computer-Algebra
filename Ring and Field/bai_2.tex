\section*{Bài 2.}

Cho $Z \big( \sqrt{-3} \big) = \big\{ a + b\sqrt{-3} : a, b \in \mathbb{Z} \big\}$. Chứng minh rằng:
\begin{enumerate}[label=\alph*.]
    \item $Z \big( \sqrt{-3} \big)$ là một miền nguyên.

    \item $1 + \sqrt{-3}$ và $1 - \sqrt{-3}$ là các phần tử bất khả quy trong $Z \big( \sqrt{-3} \big)$.\\
    Nghĩa là không tồn tại $u, v \in Z \big( \sqrt{-3} \big)$ ($u, v \neq 1$ $-$ phần tử đơn vị của phép toán "$\cdot$") để phần tử đó có thể viết dưới dạng $u \cftdot v$.
\end{enumerate}
	
\addcontentsline{toc}{section}{Bài 2}

\centering
\textbf{\underline{Bài làm:}}

\justifying
\begin{enumerate}[label=\alph*.]
    \item Dễ dàng ta thấy $Z \big( \sqrt{-3} \big)$ có nhiều hơn một phần tử.\\
    Với mọi $a_1 + b_1\sqrt{-3}, a_2 + b_2\sqrt{-3} \in Z \big( \sqrt{-3} \big)$ ta có:\\
    $\big( a_1 + b_1\sqrt{-3} \big) - \big( a_2 + b_2\sqrt{-3} \big) = \big( a_1 - a_2 \big) + \big( b_1 - b_2 \big)\sqrt{-3} \in Z \big( \sqrt{-3} \big)$\\
    $\big( a_1 + b_1\sqrt{-3} \big) \big( a_2 + b_2\sqrt{-3} \big) = \big( a_1 a_2 - 3 b_1 b_2 \big) + \big( a_1 b_2 + a_2 b_1 \big)\sqrt{-3} \in Z \big( \sqrt{-3} \big)$\\
    Vì vậy, $Z \big( \sqrt{-3} \big) \subseteq \mathbb{C}$ theo tiêu chuẩn của vành con.\\
    Mà trường số phức có tính chất giao hoán và không có ước của 0 nên $Z \big( \sqrt{-3} \big)$ cũng có tính chất giao hoán và không có ước của 0.\\
    Như vậy, $Z \big( \sqrt{-3} \big)$ là vành giao hoán có đơn vị, có nhiều hơn một phần tử và không có ước của 0 nên $Z \big( \sqrt{-3} \big)$ là miền nguyên.
    \item Với mỗi phần tử $a + b\sqrt{-3} \in Z \big( \sqrt{-3} \big)$, ta định nghĩa hàm chuẩn $N$ (norm function) của nó như sau: $N \big( a + b\sqrt{-3} \big) = \big( a + b\sqrt{-3} \big) \big( a - b\sqrt{-3} \big) = a^2 + 3b^2$.\\
    Ta có: $\big(1 + \sqrt{-3}\big)\big(1 - \sqrt{-3}\big) = 1^2 - (-3) = 4$ và $2 \cdot 2 = 4$.\\
    $N(2) = 2^2 = 4$ với $2 = uv$ ($u$, $v$ không đồng thời là phần tử đơn vị).\\
    Suy ra $N(2) = 4 = N(u)N(v)$.\\
    Điều này dẫn đến $N(u) = 2$. Nhưng điều này không thể xảy ra trên tập $\mathbb{Z}$, vì nếu $u = a + b\sqrt{-3}$ thì $N(u) = a^2 + 3b^2$, tương đương với $b = 0$ (vì $a^2$ và $3b^2$ là các số nguyên dương, nếu $b > 0$ thì $N(u) > 3$) và $a = \sqrt{2} \not \in \mathbb{Z}$.\\
    \textbf{Xét phần tử} $1 + \sqrt{-3}: N \big( 1 + \sqrt{-3} \big) = 1 + 3 = 4$.\\
    Nếu $1 + \sqrt{-3} = uv$ thì $N(u) = N(v) = 2$. Như đã chứng minh ở trên thì điều này là bất khả thi. Suy ra $1 + \sqrt{-3}$ là phần tử bất khả quy trên $Z \big( \sqrt{-3} \big)$.\\
    \textbf{Xét phần tử} $1 - \sqrt{-3}: N \big( 1 - \sqrt{-3} \big) = 1 + 3 = 4$.\\
    Chứng minh tương tự, $1 - \sqrt{-3}$ cũng là phần tử bất khả quy trên $Z \big( \sqrt{-3} \big)$.
    Như vậy, ta đã chứng minh được $1 + \sqrt{-3}$ và $1 - \sqrt{-3}$ là các phần tử bất khả quy trên $Z \big( \sqrt{-3} \big)$.
\end{enumerate}
\clearpage