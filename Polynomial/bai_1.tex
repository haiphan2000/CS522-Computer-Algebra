\section*{Bài 1.}

Cho $a < b < c$ là 3 nghiệm của đa thức:
$$P \big( x \big) = x^3 - 3x + 1$$
Chứng minh rằng: $a^2 - c = b^2 - a = c^2 - b = 2$ (*)
	
\addcontentsline{toc}{section}{Bài 1}

\centering
\textbf{\underline{Bài làm:}}

\justifying
Từ $a^2 - c = b^2 - a = c^2 - b = 2 \Rightarrow$
$ \begin{cases}
    c = a^2 - 2\\
    a = b^2 - 2\\
    b = c^2 - 2
\end{cases} $

Do đó, để chứng minh (*), ta cần chứng minh 2 điều sau:
\begin{enumerate}
    \item $a^2 - 2$, $b^2 - 2$, $c^2 -2$ là 3 nghiệm của $P(x)$.
    \item $b^2 < c^2 2 < a^2 2$ là 3 nghiệm của $P(x)$.
\end{enumerate}

Áp dụng định lý Viet cho $P(x)$ với 3 nghiệm là $a, b, c$ ta có:
$$\begin{cases}
    a + b + c = 0\\
    ab + bc + ca = -3\\
    abc = -1 
\end{cases}$$

\textbf{1.1. Chứng minh $a^2 - 2$, $b^2 - 2$, $c^2 -2$ là 3 nghiệm của $P(x)$.}

Áp dụng định lý Viet đảo, ta cần chứng minh:
$$\begin{cases}
    a^2 - 2 + b^2 - 2 + c^2 - 2 = 0\\
    (a^2 - 2)(b^2 - 2) + (b^2 -2)(c^2 - 2) + (a^2 -2)(c^2 - 2) = -3\\
    (a^2 - 2)(b^2 - 2)(c^2 - 2) = -1
\end{cases}$$

\textbf{CM1. } $a^2 - 2 + b^2 - 2 + c^2 - 2 = 0$

Ta có: 

\begin{align*}
    &a + b + c = 0 \\
    \Leftrightarrow \text{ } &(a + b + c)^2 = 0\\
    \Leftrightarrow \text{ } &a^2 + b^2 + c^2 + 2(ab + bc + ac) = 0\\
    \Leftrightarrow \text{ } &a^2 + b^2 + c^2 + 2(-3) = 0\\
    \Leftrightarrow \text{ } &a^2 + b^2 + c^2 -6 = 0\\
    \Leftrightarrow \text{ } &a^2 - 2 + b^2 - 2 + c^2 - 2 = 0
\end{align*}

\textbf{CM2. } $(a^2 - 2)(b^2 - 2) + (b^2 -2)(c^2 - 2) + (a^2 -2)(c^2 - 2) = -3$

Ta có:

\begin{align*}
    &(a^2 - 2)(b^2 - 2) + (b^2 -2)(c^2 - 2) + (a^2 -2)(c^2 - 2)\\
    = &a^2(b^2-2) - 2(b^2 -2) + b^2(c^2-2) - 2(c^2-2) + c^2(a^2-2) - 2(a^2-2)\\
    = &a^2(b^2-2) + b^2(c^2-2) + c^2(a^2-2) -2(b^2-2 + c^2-2 + a^2-2)\\
    = &a^2b^2 -2a^2 + b^2c^2 - 2b^2 + a^2c^2 -2c^2\\
    = &a^2b^2 + b^2c^2 + a^2c^2 -2(a^2 + b^2 + c^2)\\
    = &a^2b^2 + b^2c^2 + a^2c^2 -2 \cdot 6\\
    = &(ab)^2 + (bc)^2 + (ac)^2 -12\\
    = &(ab + bc + ac)^2 - 2abc(a + b + c) - 12\\
    = &(-3)^2 - 2 \cdot (-1) \cdot 0 -12\\
    = &-3
\end{align*}

\textbf{CM3. }$(a^2 - 2)(b^2 - 2)(c^2 - 2) = -1$

Ta có:

\begin{align*}
    &(a^2 - 2)(b^2 - 2)(c^2 - 2)\\
    = &a^2b^2c^2 -2a^2b^2 - 2a^2c^2 + 4a^2 -2b^2c^2 + 4b^2 + 4c^2 - 8\\
    = &(abc)^2 + 4(a^2 + b^2 + c^2) - 2((ab)^2 + (bc)^2 + (ac)^2) - 8\\
    = &(abc)^2 + 4(a^2 + b^2 + c^2) - 2[(ab + bc + ac)^2 - 2abc(a + b + c)] - 8\\
    = &(-1)^2 + 4 \cdot 6 - 2 \cdot [(-3)^2 - 2 \cdot (-1) \cdot 0] - 8\\
    = &1 + 24 - 18 - 8\\
    = &-1
\end{align*}

Do đó, ta đã chứng minh được $a^2 - 2$, $b^2 - 2$, $c^2 -2$ là 3 nghiệm của $P(x)$.

\textbf{1.2. Chứng minh $b^2 - 2 < c^2 - 2 < a^2 - 2$ là 3 nghiệm của $P(x)$.}

\begin{itemize}
    \item Đầu tiên ta có $abc = -1$ $\Leftrightarrow$
    $\begin{cases}
        a, b, c \neq 0\\
        a, b, c \text{ có 1 hoặc 3 số âm.}
    \end{cases}$
    \item Mà $a + b + c = 0$, do đó không thể xảy ra trường hợp cả 3 số đều âm, nên trong 3 số $a$, $b$, $c$ có 1 số âm.
    \item Lại có $a < b < c$, $a$ nhỏ nhất trong 3 số nên $a$ là số âm.
    \item Từ $a + b + c = 0$ $\Rightarrow$ $a = -(b + c)$ $\Rightarrow$ $a^2 = (b + c)^2$
    
    $\Rightarrow$ $a^2 = b^2 + 2bc + c^2 > b^2 + c^2$ (Vì $b, c > 0$ nên $2bc > 0$)
    
    $\Rightarrow$ $a^2 > b^2 + c^2 > c^2 > b^2$ (Vì $b < c$)
    
    $\Rightarrow$ $b^2 - 2 < c^2 - 2 < a^2 - 2$
\end{itemize}
Vậy ta chứng minh được $b^2 - 2 < c^2 - 2 < a^2 - 2$ là 3 nghiệm của $P(x)$.

\clearpage